% 
\section{Background}
\label{introduction:background}
% 
The introduction of a thesis serves as the gateway to the research, offering a concise preview of the study's core elements. It encapsulates the research problem, objectives, and significance, capturing the reader's interest. The rules for this section necessitate clarity and focus. It must succinctly present the research topic, its context, and the gaps in existing knowledge. The introduction should clearly define the scope and limitations of the study, outlining what the research aims to achieve and what it won't address. Additionally, it must establish the research's relevance, emphasizing its potential contributions to the academic field or real-world applications. A compelling introduction sets the stage, motivating readers to explore the thesis further.


% 
\subsection{Subsection(s)}
\label{introduction:background:example}
% 
Generate as many subsections as necessary to enhance the clarity and understanding of the section. While doing so, ensure a smooth flow throughout the content.


% 
\section{Motivation and Inspiration}
\label{introduction:motivation}
% 
The motivation and inspiration section elucidates the driving factors behind the research endeavor, outlining the reasons why the study is significant and necessary. It sets the context, explaining the gap in existing knowledge or the problem that the research aims to address. This section serves to captivate the reader's interest, clearly stating the research questions and objectives while demonstrating the project's relevance to the academic field or practical applications. Rules for crafting this section involve clarity in articulating the research problem, providing solid background information, and explaining the potential impact of the study. It should be concise, focusing on the core issues, and avoid vague statements. Additionally, it must establish the researcher's passion and genuine interest in the topic, compellingly conveying why the study is not only essential academically but also personally significant, thus engaging the readers emotionally and intellectually.


% 
\section{Objectives}
\label{introduction:objective}
% 
The objectives section outlines the specific goals the researcher aims to achieve. It clarifies the study's purpose, guiding the research process. The objectives must be clear, concise, and achievable within the study's scope. They should be measurable and time-bound, providing a roadmap for the research methodology. Each objective must align with the research questions, indicating what the researcher intends to accomplish. Objectives should be realistic and focused, ensuring that the study remains manageable and the outcomes contribute meaningfully to the field. Additionally, they should be framed in a way that allows for evaluation, enabling the researcher to determine if and how each objective has been met by the end of the study.


% 
\section{Scope}
\label{introduction:scope}
% 
The scope section defines the boundaries and limitations of the study, outlining what will be included and excluded. It sets the parameters for the research, clarifying the specific aspects, variables, and geographical or temporal limits under investigation. The rules for this section require precision and clarity. It must be explicit about the depth and breadth of the study, detailing the range of topics, sources, and methodologies. Avoid ambiguity, ensuring readers understand the study's focus. The scope section serves as a lens, guiding the researcher and readers about the study's scale and boundaries, providing essential context for interpreting the research findings.


% 
\section{Thesis Organization}
\label{introduction:organization}
% 
This thesis is structured into five chapters, each with a specific focus. A brief summary of these chapters is presented below.

\medskip
Chapter \ref{chapter:introduction} serves as an introduction to this thesis, highlighting its main objective, motivation, research problems, and contributions.

\medskip
Chapter \ref{chapter:literature} provides an overview of important concepts and background knowledge relevant to the domain. It also explores existing studies related to transliteration, highlighting their limitations.

\medskip
Chapter \ref{chapter:methodology} outlines the complete methodology proposed in this thesis. It covers various aspects such as dataset description, preprocessing techniques, feature encoding and selection methods, performance measures, and the models employed.

\medskip
Chapter \ref{chapter:experiment} delves into the conducted experiments and presents detailed findings. It critically evaluates and discusses the outcomes of these experiments.

\medskip
Chapter \ref{chapter:conclusion} concludes the work presented in this thesis and outlines potential directions for future research.

