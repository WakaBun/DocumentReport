% 
\section{Introduction}
\label{conclusion:introduction}
% 
The conclusion section provides a concise summary of the study's key findings, their significance, and their alignment with the research objectives. It offers closure, reaffirming the thesis statement and the contributions made to the field. Rules for this section require clarity and conciseness. It should restate the research problem and highlight the main outcomes, emphasizing their relevance. The conclusion must not introduce new ideas but rather synthesize existing ones, leaving a lasting impression on the reader. It should also address the study's limitations honestly. Lastly, the conclusion often suggests practical applications of the research and emphasizes its broader impact, leaving readers with a clear understanding of the study's significance.


% 
\section{Future Work}
\label{conclusion:future}
% 
The future works section outlines potential research directions and areas for further exploration based on the current study's findings and limitations. It suggests unresolved questions, methodologies that could enhance understanding, or applications that could arise from the research. The rules for this section necessitate insight and foresight. It should be specific, offering clear, well-defined suggestions for future researchers. Recommendations should be grounded in the study's conclusions and aimed at addressing its limitations or expanding upon its successes. Clarity is key; ideas should be presented logically, providing a roadmap for scholars interested in building upon the current research, encouraging a continuous scholarly discourse in the field.
